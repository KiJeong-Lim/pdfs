\documentclass[12pt]{paper}

\usepackage[a4paper, left=20mm, right=20mm, top=20mm, bottom=20mm]{geometry}
\usepackage[bottom]{footmisc}
\usepackage{amsmath}
\usepackage{amsfonts}
\usepackage{amsthm}
\usepackage{amssymb}
\usepackage{enumitem}
\usepackage{graphicx}
\usepackage{kotex}
\usepackage{listings}
\usepackage{mathtools}
\usepackage{pgfplots}
\usepackage{setspace}
\usepackage{tikz}

\usetikzlibrary{calc,patterns}
\pgfplotsset{compat=newest}

\title{Haskell로 만드는 인터프리터}

\author{임기정}

\newenvironment{context}[1][]{\noindent \textbf{{#1}.}}{\hfill $ \dashv $}

\newcommand{\hsstyle}
{ \lstset
  { basicstyle=\footnotesize\ttfamily
  , breakatwhitespace=false
  , breaklines=true
  , frame=single
  , keywordstyle=\color{blue}
  , morekeywords=
    { module
    , where
    , import
    , do
    , let
    , in
    , case
    , of
    }
  , tabsize=2
  }
}

\newcommand{\terminalstyle}
{ \lstset
  { basicstyle=\footnotesize\ttfamily
  , breakatwhitespace=false
  , breaklines=true
  , frame=single
  , keywordstyle=\color{blue}
  , tabsize=2
  }
}

\lstnewenvironment{hscode}[1][]
{ \setstretch{1.0}
  \hsstyle
  \lstset{#1}
}
{ \setstretch{1.5}
}

\lstnewenvironment{terminalcode}[1][]
{ \setstretch{1.0}
  \terminalstyle
  \lstset{#1}
}
{ \setstretch{1.5}
}

\begin{document}

  \nocite{seman2010}

  \setstretch{1.5}
  \maketitle \hspace{12pt}

  \section{목표} \hspace{12pt}

  다음을 프로그래밍 언어 Haskell로 구현하는 것을 목표로 한다.
  \begin{enumerate}
    \item \underline{어휘분석기 생성기}(\textit{Lexical Analyzer Generator})
    \item \underline{구문분석기 생성기}(\textit{Syntactic Analyzer Generator})
    \item Mini-C 인터프리터
  \end{enumerate}

  \section{``Hello world!''} \hspace{12pt}

  Haskell 빌드 툴인 Stack을 이용하여, ``Hello world!''를 출력해 보자.

\begin{terminalcode}
> mkdir hello
> cd hello
> vim hello.cabal
> cat hello.cabal
name: hello
version: 0.1.0
build-type: Simple
cabal-version: >= 1.10
executable main
  hs-source-dirs: src
  default-language: Haskell2010
  main-is: Main.hs
  build-depends: base >= 4.7 && < 5
  other-modules:
> mkdir src
> cd src
> vim Main.hs
> cat Main.hs
module Main where

main :: IO ()
main = putStrLn "Hello world!"

> cd ..
> stack init
> stack build hello
> stack exec -- main
Hello world!
\end{terminalcode}

  \bibliographystyle{unsrt}
  \bibliography{compilerfrontend}

\end{document}

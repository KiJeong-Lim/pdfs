\documentclass[12pt]{paper}

\usepackage{amsfonts}
\usepackage{amsmath}
\usepackage{amssymb}
\usepackage{amsthm}
\usepackage{color}
\usepackage[bottom]{footmisc}
\usepackage[a4paper, left=20mm, right=20mm, top=20mm, bottom=20mm]{geometry}
\usepackage{graphicx}
\usepackage{kotex}
\usepackage{listings}
\usepackage{setspace}
\usepackage{enumitem}

\newenvironment{context}[1][]
{ \noindent \textbf{{#1}.}
}
{ \hfill $ \dashv $
}

\newcommand{\powerset}
{ \mathcal{P}
}

\begin{document}

  \setstretch{1.5}

  \begin{context}[문]
    집합 $A$가 주어졌을 때,
    $$ \powerset \left( \bigcup A \right) \subseteq A \leftrightarrow \left( \exists a \in A \right) \left[ A = \powerset \left( a \right) \right] $$인가?
  \end{context}

  \hspace{12pt}

  \begin{context}[답]
    그렇다.
  \end{context}

  \begin{context}[Lemma 1]
    이항 관계 $R$에 대하여, $R$이 원순서일 때 그리고 그럴 때에만,
    $$ \left( \forall x \right) \left( \forall y \right) \left[ x R y \leftrightarrow \left( \forall z \right) \left( z R x \rightarrow z R y \right) \right] $$가 성립한다.
  \end{context}

  \begin{context}[Lemma 2]
    임의의 집합 $X$에 대하여 $ \bigcup \powerset \left( X \right) = X $이다.
  \end{context}

  \begin{context}[Lemma 3]
    임의의 집합 $X$에 대하여 $ \left( \forall z \in X \right) \left[ z \subseteq \bigcup X \right] $이다.
  \end{context}

  \begin{proof}
    $\subseteq$는 부분순서이므로
    보조정리 1로부터 $$ \powerset \left( \bigcup A \right) \subseteq A \iff \forall x \left[ x \subseteq \powerset \left( \bigcup A \right) \rightarrow x \subseteq A \right] $$임을 알 수 있고,
    임의의 집합 $x$에 대하여
    \begin{align*}
      x \subseteq \powerset \left( \bigcup A \right)
      & \iff \left( \forall z \in x \right) \left( z \in \powerset \left( \bigcup A \right) \right) \\
      & \iff \left( \forall z \in x \right) \left( z \subseteq \bigcup A \right) \\
      & \iff \left( \forall z \in x \right) \left( \forall y \in z \right) \left( y \in \bigcup A \right) \\
      & \iff \left( \forall y \in \bigcup x \right) \left( y \in \bigcup A \right) \\
      & \iff \bigcup x \subseteq \bigcup A
    \end{align*}이므로,
    $$ \left( \forall x \left[ \bigcup x \subseteq \bigcup A \rightarrow x \subseteq A \right] \right) \leftrightarrow \left( \exists a \in A \right) \left[ A = \powerset \left( a \right) \right] $$
    를 보이면 충분함을 알 수 있다.

    $ \left( \Leftarrow \right) $ 어떤 $a \in A$가 존재하여 $A = \powerset \left( a \right)$가 성립한다고 가정하자.
    그러면, $ \bigcup x \subseteq \bigcup A $를 만족시키는 집합 $x$가 주어졌다고 가정한 채로 $ x \subseteq A $임을 보일 차례가 된다.

    이제 $ x \subseteq A $를 보이기 위하여 $z \in x$인 집합 $z$가 주어졌다고 가정하자. 그러면
    \begin{align*}
      z \in x
      \implies z \subseteq \bigcup x
      \implies z \subseteq a
      \implies z \in \powerset \left( a \right)
      \implies z \in A
    \end{align*}
    이므로, $ \left( \Leftarrow \right) $의 증명이 끝난다.

    $ \left( \Rightarrow \right) $ 먼저 $\eqref{eq:lhs}$을 가정하자.
    \begin{equation} \label{eq:lhs}
      \forall x \left[ \bigcup x \subseteq \bigcup A \rightarrow x \subseteq A \right]
    \end{equation}
    이때 $a := \bigcup A$로 잡자. 이제 다음 명제들을 모두 보이면 된다. 
    \setstretch{1.0} \begin{itemize}
      \item [(a)] $ a \in A $.
      \item [(b)] $ \powerset \left( a \right) \subseteq A $.
      \item [(c)] $ A \subseteq \powerset \left( a \right) $.
    \end{itemize} \setstretch{1.5}
    \begin{enumerate}
      \item $\bigcup \left\{ a \right\} = a \subseteq \bigcup A$이므로,
      명제 (a)는 $\eqref{eq:lhs}$에 $x := \left\{ a \right\}$를 적용하여 얻을 수 있다.
      \item $\bigcup \powerset \left( a \right) = a \subseteq \bigcup A$이므로,
      명제 (b)는 $\eqref{eq:lhs}$에 $x := \powerset \left( a \right)$를 적용하여 얻을 수 있다.
      \item 마지막으로, 명제 (c)를 보이기 위하여, $z \in A$라고 하자.
      그러면 $$ z \subseteq \bigcup A = a $$이므로,
      원하던 대로 $z \in \powerset \left( a \right)$를 얻는다.
    \end{enumerate}
    이로써 모든 증명이 끝났다.
  \end{proof}

\end{document}
